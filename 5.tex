\setcounter{chapter}{4}
\chapter{Probabilistic Analysis and Randomized Algorithms}
%Section 5.1------------------------------------------------------------------------------------------------------------
\section{The hiring problem}
\subsection*{Exercises}
\begin{enumerate}[\thesection-1]
%
\item Recall the definition of total order. Since the order in which the candidates are presented is random, every pair of candidates can appear as candidate $1$ and candidate $2$. Use the fact that you always know whether candidate $2$ is better than candidate $1$.
%
\item Let $m \defas b - a + 1$. If $m = 2^n$ for some $n \in \pint$, then assign the binary strings over $\sete{0, 1}$ of length $n$ (say, in lexicographic order) to the numbers $a, \etl, b$; make $n$ calls to $\proc{Random}(0, 1)$ in a row. Otherwise, let $n \defas \ceil{\lg m}$, and assign the first $m$ binary strings (in lexicographic order) to the numbers $a, \etl, b$; make $n$ calls to $\proc{Random}(0, 1)$ in a row, and when it produces the remaining $2^n - m$ unassigned binary strings, perform $n$ calls to $\proc{Random}(0, 1)$ again until it produces an assigned binary string.
%
\item An algorithm is given as follows.
\begin{codebox}
\Procname{$\proc{Unbiased-Random}$}
\li $b_1 \gets \proc{Biased-Random}$
\li \If $b_1 \isequal 0$
\li \Then
          $b_2 \gets \proc{Biased-Random}$
\li       \If $b_2 \isequal 0$
\li       \Then
                \Return $\proc{Unbiased-Random}$
\li       \Else
\li             \Return $0$
\zi             \Comment{The case of $(b_1, b_2) = (0, 1)$ occurs with probability $p(1 - p)$ and is assigned to the value $0$}
          \End
\li \Else \Comment{$b_1 \isequal 1$}
\li       $b_2 \gets \proc{Biased-Random}$
\li       \If $b_2 \isequal 0$
\li       \Then
                \Return $1$
\zi             \Comment{The case of $(b_1, b_2) = (1, 0)$ occurs with probability $p(1 - p)$ and is assigned to the value $1$}
\li       \Else
\li             \Return $\proc{Unbiased-Random}$
          \End
     \End
\end{codebox}
%
\end{enumerate}
%End of Section 5.1-----------------------------------------------------------------------------------------------------

\setcounter{section}{2}
%Section 5.3------------------------------------------------------------------------------------------------------------
\section{Randomized algorithms}
\begin{enumerate}[\thesection-1]
%
\item Unroll the \For loop of $\proc{Randomly-Permute}(A, n)$ for $i = 1$.
\begin{codebox}
\Procname{$\proc{Randomly-Permute}'(A, n)$}
\li swap $A[1]$ with $A[\proc{Random}(1, n)]$
\li \For $i \gets 2$ \To $n$
\li \Do
        swap $A[i]$ with $A[\proc{Random}(i, n)]$
    \End
\end{codebox}
The associated loop invariant is below.
\begin{loopinv}
Just prior to the $(i - 1)$st iteration of the \For loop of lines $2$-$3$, for each possible $(i - 1)$-permutation of the $n$ elements, the subarray $A[1 \subarr i - 1]$ contains this $(i - 1)$-permutation with probability $(n - i + 1)!/n!$.
\end{loopinv}
%
\setcounter{enumi}{2}
%
\item Let $n \geq 3$. The procedure $\proc{Permute-With-All}(A, n)$ does not produce a uniform random permutation. To see this, consider a complete $n$-ary rooted tree of height $n$ in which the $n$ children of each node at depth $(i - 1)$, where $i = 1, \etl, n$, correspond to the $n$ possible outcomes produced by the call to $\proc{Random}(1, n)$ in the $i$th iteration of the \For loop of $\proc{Permute-With-All}(A, n)$; thus, the leaves give exactly the possible outcomes by executing $\proc{Permute-With-All}(A, n)$. Note that there are $n^n$ leaves, while the total number of $n$-permutations of $A$ is $n!$. Since $n - 1$ does not divide $n$, neither does $n!$ divide $n^n$. Consequently, some $n$-permutations are more likely than others.
%
\setcounter{enumi}{4}
%
\item Prove the loop invariant below.
\begin{loopinv}
Just prior to the $(k - n + m)$th iteration of the \For loop of lines $2$-$6$, every $(k - n + m - 1)$-combination of the set $\sete{1, \etl, k - 1}$ is equally likely to be the set $S$, with probability $1/{k - 1 \choose k - n + m - 1}$.
\end{loopinv}
In the maintenance part, distinguish two cases based on whether or not it holds that $i \in S$ or $i = k$ immediately after the call to $\proc{Random}(i, k)$ on line $3$ of the procedure $\proc{Random-Sample}(m, n)$.

\begin{remark}
Like many methods based on randomness (e.g., Exercise 5.1-3), the procedure $\proc{Random-Sample}(m, n)$ works by ``interpreting'' the outcomes of executing a random procedure to achieve a desired distribution. A similar idea that kind of goes the other way around is the following counting argument to show that the number of pairs $(a, b)$, where $a \leq b$ are from the set $\sete{1, \etl, n}$, is equal to ${n + 1 \choose 2}$: The number of such pairs is equal to the number of $2$-combinations of the set $\sete{1, \etl, n, \mbox{``duplicate''}}$, where for each $k \in \sete{1, \etl, n}$ the $2$-combination $\sete{i, \mbox{``duplicate''}}$ is interpreted as the pair $(k, k)$.
\end{remark}
%
\end{enumerate}
%End of Section 5.3-----------------------------------------------------------------------------------------------------

%Section 5.4------------------------------------------------------------------------------------------------------------
\section{Probabilistic analysis and further uses of indicator random variables}
\begin{enumerate}[\thesection-1]
\setcounter{enumi}{2}
%
\item Note how this exercise question is similar to the birthday paradox presented in Subsection 5.4.1~in text. For $1 \leq k \leq b + 1$ let $A_k$ be the event that the $k$th toss ends up in an empty bin, and for $1 \leq k \leq b$ let $B_k$ be the event that the first $k$ tosses end up in an empty bin each (these events are similar to those given in 5.4.1). We have $\pr{A_1} = \pr{B_1} = 1$ and $\pr{A_{b + 1}} = 0$; more importantly, for $2 \leq k \leq b$, we have
\[
\pr{A_k \given B_{k - 1}} = \parenf{1 - \frac{k - 1}{b}}
\]
and hence
\[
\begin{array}{lcl}
\pr{B_k} & = & \pr{B_{k - 1}} \pr{A_k \given B_{k - 1}} \cr
         & \etv & \cr
         & = & \pr{B_1} \pr{A_2 \given A_1} \pr{A_3 \given B_2} \etc \pr{A_k \given B_{k - 1}} \cr
         & = & \ds{1 \multp \parenf{1 - \frac{1}{b}} \parenf{1 - \frac{2}{b}} \etc \parenf{1 - \frac{k - 1}{b}}}. \cr
\end{array}
\]
because $B_k = A_k \intsc B_{k - 1}$.

Let $X$ be the number of tosses until some bin contains two balls. Then, for $2 \leq k \leq b + 1$ we have
\[
\pr{X = k} = \pr{\cmpl{A_k} \intsc B_{k - 1}} = \pr{B_{k - 1}} \pr{\cmpl{A_k} \given B_{k - 1}} = 1 \multp \parenf{1 - \frac{1}{b}} \parenf{1 - \frac{2}{b}} \etc \parenf{1 - \frac{k - 2}{b}} \multp \frac{k - 1}{b}.
\]
It suffices to compute
\[
\ex{X} = \sum^{b + 1}_{k = 2} k \pr{X = k}.
\]
%
\item The birthdays are mutually independent. On page 141, the probability that all $k$ people in the room have distinct birthdays is
\[
\pr{B_k} = 1 \multp \parenf{1 - \frac{1}{n}} \parenf{1 - \frac{2}{n}} \etc \parenf{1 - \frac{k - 1}{n}}.
\]
Since there are ${n \choose k} k!$ different cases in which all the $k$ people have distinct birthdays, namely $\card{B_k} = {n \choose k} k!$, and since each case is equally likely due to the uniform distribution of birthdays, we have that, for all distinct numbers $n_1, \etl, n_k$ in the range $1, \etl, n$, the probability
\[
\pr{b_1 = n_1, \etl, b_k = n_k} = \frac{\pr{B_k}}{{n \choose k} k!} = \frac{1}{n^k}.
\]
This means that it is implicitly assumed that the birthdays $b_1, \etl, b_k$ are mutually independent in the analysis.
%
\item Assume that there are $n = 365$ days in a year and that there are $k$ people in a room. Let $E_m$ be the event that among the $k$ people, there are exactly $m$ pairs of people sharing the same birthday, each pair a distinct birthday. Then
\[
\card{E_0} = {n \choose k} \multp k!
\]
and
\[
\card{E_m} = \frac{{k \choose 2}{k - 2 \choose 2} \etc {k - 2m + 2 \choose 2}}{m!} \multp {n \choose k - m} \multp (k - m)!
\]
for $m \geq 1$. Hence,
\[
\pr{E_m} = \frac{\card{E_m}}{n^k}
\]
for $m \geq 0$ since the birthdays are mutually independent. The probability that there are at least three people sharing the same birthday is
\[
1 - \sum^{\floor{\frac{k}{2}}}_{m = 0} \pr{E_m}.
\]
%
\item The probability that a $k$-string over a set of size $n$ forms a $k$-permutation is $\frac{{n \choose k} k!}{n^k}$. This is analogous to the birthday paradox in that there are $n$ days in a year and $k$ people in a room.
%
\item For $1 \leq i \leq n$, let $X_i$ and $Y_i$ be indicator random variables such that
\[
X_i = \ind{\mbox{the \mathmode{i}th bin is empty}},
\]
and
\[
Y_i = \ind{\mbox{the \mathmode{i}th bin contains exactly one ball}}.
\]
then we have
\[
\pr{X_i = 1} = \frac{(n - 1)^n}{n^n} = \parenf{1 - \frac{1}{n}}^n
\]
and
\[
\pr{Y_i = 1} = \frac{{n \choose 1}(n - 1)^{n - 1}}{n^n} = \parenf{1 - \frac{1}{n}}^{n - 1}.
\]

Thus,
\[
\ex{\sum^n_{i = 1} X_i} = \sum^n_{i = 1} \ex{X_i} = n \multp \parenf{1 - \frac{1}{n}}^n
\]
and
\[
\ex{\sum^n_{i = 1} Y_i} = \sum^n_{i = 1} \ex{Y_i} = n \multp \parenf{1 - \frac{1}{n}}^{n - 1}.
\]
%
\end{enumerate}
%End of Section 5.4-----------------------------------------------------------------------------------------------------

%Problems of Chapter 5--------------------------------------------------------------------------------------------------
%\section*{Problems}
%\begin{enumerate}[\thechapter-1]
%
%\item 
%
%\end{enumerate}