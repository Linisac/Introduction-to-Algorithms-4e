\setcounter{chapter}{4}
\chapter{Probabilistic Analysis and Randomized Algorithms}
%Section 5.1-----------------------------------------------------------------------
\section{The hiring problem}
\subsection*{Exercises}
\begin{enumerate}[\thesection-1]
%
\item Recall the definition of total order. Since the order in which the candidates are presented is random, every pair of candidates can appear as candidate $1$ and candidate $2$. Use the fact that you always know whether candidate $2$ is better than candidate $1$.
%
\item Let $m \defas b - a + 1$. If $m = 2^n$ for some $n \in \pint$, then assign the binary strings over $\sete{0, 1}$ of length $n$ (say, in lexicographic order) to the numbers $a, \etl, b$; make $n$ calls to $\proc{Random}(0, 1)$ in a row. Otherwise, let $n \defas \ceil{\lg m}$, and assign the first $m$ binary strings (in lexicographic order) to the numbers $a, \etl, b$; make $n$ calls to $\proc{Random}(0, 1)$ in a row, and when it produces the remaining $2^n - m$ unassigned binary strings, perform $n$ calls to $\proc{Random}(0, 1)$ again until it produces an assigned binary string.
%
\item An algorithm is given as follows.
\begin{codebox}
\Procname{$\proc{Unbiased-Random}$}
\li $b_1 \gets \proc{Biased-Random}$
\li \If $b_1 \isequal 0$
\li \Then
          $b_2 \gets \proc{Biased-Random}$
\li       \If $b_2 \isequal 0$
\li       \Then
                \Return $\proc{Unbiased-Random}$
\li       \Else
\li             \Return $0$
\zi             \Comment{The case of $(b_1, b_2) = (0, 1)$ occurs with probability $p(1 - p)$ and is assigned to the value $0$}
          \End
\li \Else \Comment{$b_1 \isequal 1$}
\li       $b_2 \gets \proc{Biased-Random}$
\li       \If $b_2 \isequal 0$
\li       \Then
                \Return $1$
\zi             \Comment{The case of $(b_1, b_2) = (1, 0)$ occurs with probability $p(1 - p)$ and is assigned to the value $1$}
\li       \Else
\li             \Return $\proc{Unbiased-Random}$
          \End
     \End
\end{codebox}
%
\end{enumerate}
%End of Section 5.1----------------------------------------------------------------

\setcounter{section}{2}
%Section 5.3-----------------------------------------------------------------------
\section{Randomized algorithms}
\begin{enumerate}[\thesection-1]
%
\item Unroll the \For loop of $\proc{Randomly-Permute}(A, n)$ for $i = 1$.
\begin{codebox}
\Procname{$\proc{Randomly-Permute}'(A, n)$}
\li swap $A[1]$ with $A[\proc{Random}(1, n)]$
\li \For $i \gets 2$ \To $n$
\li \Do
        swap $A[i]$ with $A[\proc{Random}(i, n)]$
    \End
\end{codebox}
The associated loop invariant is below.
\begin{loopinv}
Just prior to the $(i - 1)$st iteration of the \For loop of lines $2$-$3$, for each possible $(i - 1)$-permutation of the $n$ elements, the subarray $A[1 \subarr i - 1]$ contains this $(i - 1)$-permutation with probability $(n - i + 1)!/n!$.
\end{loopinv}
%
\setcounter{enumi}{2}
%
\item Let $n \geq 3$. The procedure $\proc{Permute-With-All}(A, n)$ does not produce a uniform random permutation. To see this, consider a complete $n$-ary rooted tree of height $n$ in which the $n$ children of each node at depth $(i - 1)$, where $i = 1, \etl, n$, correspond to the $n$ possible outcomes produced by the call to $\proc{Random}(1, n)$ in the $i$th iteration of the \For loop of $\proc{Permute-With-All}(A, n)$; thus, the leaves give exactly the possible outcomes by executing $\proc{Permute-With-All}(A, n)$. Note that there are $n^n$ leaves, while the total number of $n$-permutations of $A$ is $n!$. Since $n - 1$ does not divide $n$, neither does $n!$ divide $n^n$. Consequently, some $n$-permutations are more likely than others.
%
\setcounter{enumi}{4}
%
\item Prove the loop invariant below.
\begin{loopinv}
Just prior to the $(k - n + m)$th iteration of the \For loop of lines $2$-$6$, every $(k - n + m - 1)$-combination of the set $\sete{1, \etl, k - 1}$ is equally likely to be the set $S$, with probability $1/{k - 1 \choose k - n + m - 1}$.
\end{loopinv}
In the maintenance part, distinguish two cases based on whether or not it holds that $i \in S$ or $i = k$ immediately after the call to $\proc{Random}(i, k)$ on line $3$ of the procedure $\proc{Random-Sample}(m, n)$.

\begin{remark}
Like many methods based on randomness (e.g., Exercise 5.1-3), the procedure $\proc{Random-Sample}(m, n)$ works by ``interpreting'' the outcomes of executing a random procedure to achieve a desired distribution. A similar idea that kind of goes the other way around is the following counting argument to show that the number of pairs $(a, b)$, where $a \leq b$ are from the set $\sete{1, \etl, n}$, is equal to ${n + 1 \choose 2}$: The number of such pairs is equal to the number of $2$-combinations of the set $\sete{1, \etl, n, \mbox{``duplicate''}}$, where for each $k \in \sete{1, \etl, n}$ the $2$-combination $\sete{i, \mbox{``duplicate''}}$ is interpreted as the pair $(k, k)$.
\end{remark}
%
\end{enumerate}
%End of Section 5.3----------------------------------------------------------------

%Problems of Chapter 5-------------------------------------------------------------
\section*{Problems}
\begin{enumerate}[\thechapter-1]
%
\item Skipped.
%
\end{enumerate}