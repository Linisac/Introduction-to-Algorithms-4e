%Last modified by Linisac, 03:37AM 04/11/2016
\documentclass{report}

%Packages
\usepackage{titlesec}
\RequirePackage{clrscode4e}
\usepackage{amssymb}
\usepackage{amsmath}
\usepackage{amsthm}
\usepackage{enumerate}
\usepackage{paralist}
\usepackage{colonequals}
\usepackage{ifthen}
\usepackage{appendix}
\usepackage[margin=1.5cm]{geometry}

%New oommands
%%Styles
\newcommand{\textib}[1]{\textit{\textbf{#1}}}
\newcommand{\df}[2]{{\displaystyle\frac{#1}{#2}}}
\newcommand{\ds}[1]{{\displaystyle#1}}
\newcommand{\paren}[1]{{\rm(}#1{\rm)}} %straight parenthese
\newcommand{\parenf}[1]{\left(#1\right)} %flexible parentheses
\newcommand{\of}{\parenf}
\newcommand{\brackf}[1]{\left[#1\right]} %flexible brackets
\newcommand{\ofb}{\brackf}
\newcommand{\curlyf}[1]{\left\{#1\right\}} %flexble curly braces
\newcommand{\ofc}{\curlyf}
\newcommand{\mathmode}[1]{\begin{math}#1\end{math}} %math mode inside text mode nested in math mode
\newcommand{\mathtext}[1]{\text{\rm#1}} %text mode inside math mode
%%Relational symbols
\newcommand{\defas}{\colonequals} %defined as
%%Mathematical constants
\newcommand{\e}{\mathit{e}} %Euler's number
%%Enumeration, sequence, chain and serial
\newcommand{\seqe}[2]{\left\langle\enum{#1}{#2}\right\rangle} %sequence by enumeration: \seq{a}{n} \gives \langle a_1, a_2, \ldots, a_n \rangle
\newcommand{\etl}{\ldots} %etc, lower
\newcommand{\etc}{\cdots} %etc, centered
\newcommand{\etv}{\vdots} %etc, vertical
%%Mathematical operators notations
\newcommand{\multp}{\cdot} %multiplication
%%Integer functions
\newcommand{\floor}[1]{\left\lfloor #1 \right\rfloor}
\newcommand{\ceil}[1]{\left\lceil #1 \right\rceil}
%%Asymptotic functions notations
\newcommand{\bigohf}{\mathit{O}} %big oh function symbol
\newcommand{\bigoh}[1]{\bigohf\of{#1}} %asymptotic big oh
\newcommand{\littleohf}{\mathit{o}} %little oh function symbol
\newcommand{\littleoh}[1]{\littleohf\of{#1}} %asymptotic little oh
\newcommand{\bigtheta}[1]{\Theta\of{#1}} %asymptotic theta
\newcommand{\bigomega}[1]{\Omega\of{#1}} %asymptotic big omega
\newcommand{\littleomega}[1]{\omega\of{#1}} %asymptotic little omega
%%Set-theoretic notations
\newcommand{\set}[1]{\ofc{#1}} %prototype of set
\newcommand{\sete}{\set} %set by enumeration
\newcommand{\suchthat}{:}
\newcommand{\setm}[2]{\set{#1 \suchthat #2}} %set by math description
\newcommand{\sett}[2]{\set{#1 \suchthat \mbox{#2}}} %set by text description
\newcommand{\intsc}{\cap} %intersection
\newcommand{\union}{\cup}
\newcommand{\cmpl}[1]{\overline{#1}} %complement
\newcommand{\nat}{\mathbb{N}} %the set of natural numbers
\newcommand{\zah}{\mathbb{Z}} %the set of integers
\newcommand{\pint}{{\zah^+}} %the set of positive integers
\newcommand{\rat}{\mathbb{Q}} %the set of rational numbers
\newcommand{\real}{\mathbb{R}} %the set of real numbers
\newcommand{\cart}{\times} %cartesian product
\newcommand{\cartpwr}[2]{#1^{#2}} %cartesian power
\newcommand{\card}[1]{\left|#1\right|} %cardinality
\newcommand{\inv}[1]{#1^{-1}} %inverse (of a function)
%%Probability-theoretic notations
\newcommand{\ind}[1]{\mathrm{I}\ofc{#1}} %indicator random variable
\newcommand{\given}{\mid}
\newcommand{\pr}[1]{\Pr\ofc{#1}} %probability
\newcommand{\exf}{\mathrm{E}} %expectation function symbol
\newcommand{\ex}[1]{\exf\ofb{#1}} %expectation
\newcommand{\var}[1]{\mathrm{Var}\ofb{#1}} %variation
\newcommand{\bindist}{\mathit{b}} %binomial distribution
%%Graph-theoretic notations
\newcommand{\degree}[1]{\mathrm{degree}\of{#1}}

%New environment
\newcommand{\Input}{\item[Input:]} %ingredient of the new environment `problem'
\newcommand{\Output}{\item[Output:]} %ingredient of the new environment `problem'
\newenvironment{problem}{\begin{description}}{\end{description}} %specification of a computational problem
\newenvironment{loopinv}{\begin{quote}}{\end{quote}} %loop invariant
\newcommand{\Init}{\item[Initialization:]} %ingredient of the new environment `loopinvpf'
\newcommand{\Main}{\item[Maintenance:]} %ingredient of the new environment `loopinvpf'
\newcommand{\Term}{\item[Termination:]} %ingredient of the new environment `loopinvpf'
\newenvironment{loopinvpf}{\begin{description}}{\end{description}} %proof of loop invariant
\theoremstyle{remark}
\newtheorem*{remark}{Remark}

%Custom settings
\titleformat{\chapter}[hang]{\huge\bfseries}{\thechapter}{.5em}{}
\renewcommand{\thesubsection}{}

\begin{document}
%\noindent
%\textsc{\Huge Introduction to\\Algorithms}
%\\
%\textsc{\Large - Annotations and Solutions Manual -}
%\\
%\\
\setcounter{chapter}{4}
\chapter{Probabilistic Analysis and Randomized Algorithms}
%Section 5.1------------------------------------------------------------------------------------------------------------
\section{The hiring problem}
\subsection*{Exercises}
\begin{enumerate}[\thesection-1]
%
\item Recall the definition of total order. Since the order in which the candidates are presented is random, every pair of candidates can appear as candidate $1$ and candidate $2$. Use the fact that you always know whether candidate $2$ is better than candidate $1$.
%
\item Let $m \defas b - a + 1$. If $m = 2^n$ for some $n \in \pint$, then assign the binary strings over $\sete{0, 1}$ of length $n$ (say, in lexicographic order) to the numbers $a, \etl, b$; make $n$ calls to $\proc{Random}(0, 1)$ in a row. Otherwise, let $n \defas \ceil{\lg m}$, and assign the first $m$ binary strings (in lexicographic order) to the numbers $a, \etl, b$; make $n$ calls to $\proc{Random}(0, 1)$ in a row, and when it produces the remaining $2^n - m$ unassigned binary strings, perform $n$ calls to $\proc{Random}(0, 1)$ again until it produces an assigned binary string.
%
\item An algorithm is given as follows.
\begin{codebox}
\Procname{$\proc{Unbiased-Random}$}
\li $b_1 \gets \proc{Biased-Random}$
\li \If $b_1 \isequal 0$
\li \Then
          $b_2 \gets \proc{Biased-Random}$
\li       \If $b_2 \isequal 0$
\li       \Then
                \Return $\proc{Unbiased-Random}$
\li       \Else
\li             \Return $0$
\zi             \Comment{The case of $(b_1, b_2) = (0, 1)$ occurs with probability $p(1 - p)$ and is assigned to the value $0$}
          \End
\li \Else \Comment{$b_1 \isequal 1$}
\li       $b_2 \gets \proc{Biased-Random}$
\li       \If $b_2 \isequal 0$
\li       \Then
                \Return $1$
\zi             \Comment{The case of $(b_1, b_2) = (1, 0)$ occurs with probability $p(1 - p)$ and is assigned to the value $1$}
\li       \Else
\li             \Return $\proc{Unbiased-Random}$
          \End
     \End
\end{codebox}
%
\end{enumerate}
%End of Section 5.1-----------------------------------------------------------------------------------------------------

\setcounter{section}{2}
%Section 5.3------------------------------------------------------------------------------------------------------------
\section{Randomized algorithms}
\begin{enumerate}[\thesection-1]
%
\item Unroll the \For loop of $\proc{Randomly-Permute}(A, n)$ for $i = 1$.
\begin{codebox}
\Procname{$\proc{Randomly-Permute}'(A, n)$}
\li swap $A[1]$ with $A[\proc{Random}(1, n)]$
\li \For $i \gets 2$ \To $n$
\li \Do
        swap $A[i]$ with $A[\proc{Random}(i, n)]$
    \End
\end{codebox}
The associated loop invariant is below.
\begin{loopinv}
Just prior to the $(i - 1)$st iteration of the \For loop of lines $2$-$3$, for each possible $(i - 1)$-permutation of the $n$ elements, the subarray $A[1 \subarr i - 1]$ contains this $(i - 1)$-permutation with probability $(n - i + 1)!/n!$.
\end{loopinv}
%
\setcounter{enumi}{2}
%
\item Let $n \geq 3$. The procedure $\proc{Permute-With-All}(A, n)$ does not produce a uniform random permutation. To see this, consider a complete $n$-ary rooted tree of height $n$ in which the $n$ children of each node at depth $(i - 1)$, where $i = 1, \etl, n$, correspond to the $n$ possible outcomes produced by the call to $\proc{Random}(1, n)$ in the $i$th iteration of the \For loop of $\proc{Permute-With-All}(A, n)$; thus, the leaves give exactly the possible outcomes by executing $\proc{Permute-With-All}(A, n)$. Note that there are $n^n$ leaves, while the total number of $n$-permutations of $A$ is $n!$. Since $n - 1$ does not divide $n$, neither does $n!$ divide $n^n$. Consequently, some $n$-permutations are more likely than others.
%
\setcounter{enumi}{4}
%
\item Prove the loop invariant below.
\begin{loopinv}
Just prior to the $(k - n + m)$th iteration of the \For loop of lines $2$-$6$, every $(k - n + m - 1)$-combination of the set $\sete{1, \etl, k - 1}$ is equally likely to be the set $S$, with probability $1/{k - 1 \choose k - n + m - 1}$.
\end{loopinv}
In the maintenance part, distinguish two cases based on whether or not it holds that $i \in S$ or $i = k$ immediately after the call to $\proc{Random}(i, k)$ on line $3$ of the procedure $\proc{Random-Sample}(m, n)$.

\begin{remark}
Like many methods based on randomness (e.g., Exercise 5.1-3), the procedure $\proc{Random-Sample}(m, n)$ works by ``interpreting'' the outcomes of executing a random procedure to achieve a desired distribution. A similar idea that kind of goes the other way around is the following counting argument to show that the number of pairs $(a, b)$, where $a \leq b$ are from the set $\sete{1, \etl, n}$, is equal to ${n + 1 \choose 2}$: The number of such pairs is equal to the number of $2$-combinations of the set $\sete{1, \etl, n, \mbox{``duplicate''}}$, where for each $k \in \sete{1, \etl, n}$ the $2$-combination $\sete{i, \mbox{``duplicate''}}$ is interpreted as the pair $(k, k)$.
\end{remark}
%
\end{enumerate}
%End of Section 5.3-----------------------------------------------------------------------------------------------------

%Section 5.4------------------------------------------------------------------------------------------------------------
\section{Probabilistic analysis and further uses of indicator random variables}
\begin{enumerate}[\thesection-1]
\setcounter{enumi}{2}
%
\item Note how this exercise question is similar to the birthday paradox presented in Subsection 5.4.1~in text. For $1 \leq k \leq b + 1$ let $A_k$ be the event that the $k$th toss ends up in an empty bin, and for $1 \leq k \leq b$ let $B_k$ be the event that the first $k$ tosses end up in an empty bin each (these events are similar to those given in 5.4.1). We have $\pr{A_1} = \pr{B_1} = 1$ and $\pr{A_{b + 1}} = 0$; more importantly, for $2 \leq k \leq b$, we have
\[
\pr{A_k \given B_{k - 1}} = \parenf{1 - \frac{k - 1}{b}}
\]
and hence
\[
\begin{array}{lcl}
\pr{B_k} & = & \pr{B_{k - 1}} \pr{A_k \given B_{k - 1}} \cr
         & \etv & \cr
         & = & \pr{B_1} \pr{A_2 \given A_1} \pr{A_3 \given B_2} \etc \pr{A_k \given B_{k - 1}} \cr
         & = & \ds{1 \multp \parenf{1 - \frac{1}{b}} \parenf{1 - \frac{2}{b}} \etc \parenf{1 - \frac{k - 1}{b}}}. \cr
\end{array}
\]
because $B_k = A_k \intsc B_{k - 1}$.

Let $X$ be the number of tosses until some bin contains two balls. Then, for $2 \leq k \leq b + 1$ we have
\[
\pr{X = k} = \pr{\cmpl{A_k} \intsc B_{k - 1}} = \pr{B_{k - 1}} \pr{\cmpl{A_k} \given B_{k - 1}} = 1 \multp \parenf{1 - \frac{1}{b}} \parenf{1 - \frac{2}{b}} \etc \parenf{1 - \frac{k - 2}{b}} \multp \frac{k - 1}{b}.
\]
It suffices to compute
\[
\ex{X} = \sum^{b + 1}_{k = 2} k \pr{X = k}.
\]
%
\item The birthdays are mutually independent. On page 141, the probability that all $k$ people in the room have distinct birthdays is
\[
\pr{B_k} = 1 \multp \parenf{1 - \frac{1}{n}} \parenf{1 - \frac{2}{n}} \etc \parenf{1 - \frac{k - 1}{n}}.
\]
Since there are ${n \choose k} k!$ different cases in which all the $k$ people have distinct birthdays, namely $\card{B_k} = {n \choose k} k!$, and since each case is equally likely due to the uniform distribution of birthdays, we have that, for all distinct numbers $n_1, \etl, n_k$ in the range $1, \etl, n$, the probability
\[
\pr{b_1 = n_1, \etl, b_k = n_k} = \frac{\pr{B_k}}{{n \choose k} k!} = \frac{1}{n^k}.
\]
This means that it is implicitly assumed that the birthdays $b_1, \etl, b_k$ are mutually independent in the analysis.
%
\item Assume that there are $n = 365$ days in a year and that there are $k$ people in a room. Let $E_m$ be the event that among the $k$ people, there are exactly $m$ pairs of people sharing the same birthday, each pair a distinct birthday. Then
\[
\card{E_0} = {n \choose k} \multp k!
\]
and
\[
\card{E_m} = \frac{{k \choose 2}{k - 2 \choose 2} \etc {k - 2m + 2 \choose 2}}{m!} \multp {n \choose k - m} \multp (k - m)!
\]
for $m \geq 1$. Hence,
\[
\pr{E_m} = \frac{\card{E_m}}{n^k}
\]
for $m \geq 0$ since the birthdays are mutually independent. The probability that there are at least three people sharing the same birthday is
\[
1 - \sum^{\floor{\frac{k}{2}}}_{m = 0} \pr{E_m}.
\]
%
\item The probability that a $k$-string over a set of size $n$ forms a $k$-permutation is $\frac{{n \choose k} k!}{n^k}$. This is analogous to the birthday paradox in that there are $n$ days in a year and $k$ people in a room.
%
\item For $1 \leq i \leq n$, let $X_i$ and $Y_i$ be indicator random variables such that
\[
X_i = \ind{\mbox{the \mathmode{i}th bin is empty}},
\]
and
\[
Y_i = \ind{\mbox{the \mathmode{i}th bin contains exactly one ball}}.
\]
then we have
\[
\pr{X_i = 1} = \frac{(n - 1)^n}{n^n} = \parenf{1 - \frac{1}{n}}^n
\]
and
\[
\pr{Y_i = 1} = \frac{{n \choose 1}(n - 1)^{n - 1}}{n^n} = \parenf{1 - \frac{1}{n}}^{n - 1}.
\]

Thus,
\[
\ex{\sum^n_{i = 1} X_i} = \sum^n_{i = 1} \ex{X_i} = n \multp \parenf{1 - \frac{1}{n}}^n
\]
and
\[
\ex{\sum^n_{i = 1} Y_i} = \sum^n_{i = 1} \ex{Y_i} = n \multp \parenf{1 - \frac{1}{n}}^{n - 1}.
\]
%
\end{enumerate}
%End of Section 5.4-----------------------------------------------------------------------------------------------------

%Problems of Chapter 5--------------------------------------------------------------------------------------------------
%\section*{Problems}
%\begin{enumerate}[\thechapter-1]
%
%\item 
%
%\end{enumerate}
\begin{appendices}
%\setcounter{chapter}{2}
\chapter{Counting and Probability}
%Section C.1------------------------------------------------------------------------------------------------------------
\section{Counting}
\subsection*{Exercises}
\begin{enumerate}[\thesection-1]
%
\item For $1 \leq k \leq n$, the number of $k$-substrings of an $n$-string is $(n - k + 1)$; there is one $0$-substring of an $n$-string, namely the empty string. Therefore, an $n$-string has in total
\[
1 + \sum^n_{k = 1} (n - k + 1) = 1 + \frac{1}{2}n(n + 1)
\]
substrings.

Note that the number of nonempty substrings is equal to ${n + 1 \choose 2}$, and it can be ``explained'' by this counting argument: Every nonempty substring can be identified by the smallest and the greatest positions, namely two (not necessarily distinct) numbers $i$ and $j$ from the set $\sete{1, \etl, n}$; the total number of combinations of $i$ and $j$ is equal to the number of $2$-combinations of the set $\sete{1, \etl, n, \mbox{``duplicate''}}$, where $\sete{k, \mbox{``duplicate''}}$ is interpreted as $i = j = k$.
%
\setcounter{enumi}{4}
%
\item Consider the $k$-combinations of an $n$-set, in which exactly one of the $k$ chosen member is a designated \emph{leader}. Either
\begin{inparaenum}[(1)]
%%
\item first choose the $k$-combination (including the leader) and then choose one of the $k$ members as the leader, or
%%
\item first choose the leader and then choose the remaining $k - 1$ non-leader members.
%%
\end{inparaenum}
%
\item Consider the $(k + 1)$-combinations of an $n$-set, in which exactly one of the $k + 1$ chosen member is a designated \emph{leader}. Either
\begin{inparaenum}[(1)]
%%
\item first choose the $k$ non-leader members and then choose the leader from the remaining $n - k$ objects (not chosen earlier), or
%
\item first choose the leader and then choose the $k$ members from the remaining $n - 1$ objects.
%%
\end{inparaenum}
%
\setcounter{enumi}{8}
%
\item Consider the number of nonempty substrings of an $n$-string: Each substring can be identified by the leftmost position and the rightmost position.
%
\item Assume that $n \geq 2$. The expression ${n \choose k}$ achieves its locally maximum value when ${n \choose k} / {n \choose k - 1} \geq 1$ and ${n \choose k + 1} / {n \choose k} \geq 1$ hold simultaneously.
%
\item Combinatorial argument: choose $j + k$ items out of $n$ vs.~choose $j + k$ items out of $n$ in which $j$ are assigned a label.
%
\item Skipped.
%
\setcounter{enumi}{14}
%
\item Ignore $k = 0$ on the left hand side, i.e., assume $k \geq 1$. Form a nonempty group of people out of $n$, in which there is a \emph{leader}. The left side is calculated by the size of the group, while the right side is calculated by asking, after the leader has been chosen, whether each of the remaining $n - 1$ people is to be counted in or not.
%
\item Skipped.
%
\end{enumerate}
%End of Section C.1-----------------------------------------------------------------------------------------------------

%Section C.2------------------------------------------------------------------------------------------------------------
\section{Probability}
\subsection*{Exercises}
\begin{enumerate}[\thesection-1]
%
\setcounter{enumi}{1}
%
\item Let $A_1, A_2, \etl$ be a countably infinite sequence of events. Observe that
\begin{enumerate}[(1)]
%%
\item $\pr{A_{i + 1}} = \pr{(A_1 \union \etc \union A_i) \intsc A_{i + 1}} + \pr{(\cmpl{A_1} \intsc \etc \intsc \cmpl{A_i}) \intsc A_{i + 1}}$ for all $i \geq 1$ (by axiom 3),
%%
\item $A_1 \union A_2 \union \etc = A_1 \union (\cmpl{A_1} \intsc A_2) \union \etc$.
%%
\end{enumerate}
By (1) and axiom 1, we have that
\begin{enumerate}[(1)]
%
\setcounter{enumii}{2}
%%
\item $0 \leq \pr{(\cmpl{A_1} \intsc \etc \intsc \cmpl{A_{i - 1}}) \intsc A_i} \leq \pr{A_i}$ for all $i \geq 2$.
%%
\end{enumerate}
Thus,
\[
\begin{array}{lll}
     & \pr{A_1 \union A_2 \union \etc} & \cr
=    & \pr{A_1 \union (\cmpl{A_1} \intsc A_2) \union \etc} & \mbox{(by (2))} \cr
=    & \ds{\pr{A_1} + \sum^\infty_{i = 2} \pr{(\cmpl{A_1} \intsc \etc \intsc \cmpl{A_{i - 1}}) \intsc A_i}} & \mbox{(by axiom 3)} \cr
\leq & \ds{\pr{A_1} + \sum^\infty_{i = 2} \pr{A_i}} & \mbox{(by (3))} \cr
=    & \pr{A_1} + \pr{A_2} + \etc. & \cr
\end{array}
\]
%
\setcounter{enumi}{4}
%
\item By induction on $n \geq 2$.
%
\item Skipped.
%
\item Let $S = \sete{(0, 0, 0), (0, 1, 0), (1, 0, 0), (1, 1, 0), (0, 0, 1)}$, $A = \setm{(x, y, z)}{x = 0}$, $B = \setm{(x, y, z)}{y = 0}$ and $C = \setm{(x, y, z)}{z = 0}$.
%
\item Consider the following list of symbols for different events.\\
\begin{tabular}{lll}
$J$  & for & ``Jeff will pass,'' \cr
$T$  & for & ``Tim will pass,'' \cr
$C$  & for & ``Carmine will pass,'' \cr
$J'$ & for & ``Professor Gore tells Carmine that Jeff will fail,'' \cr
$T'$ & for & ``Professor Gore tells Carmine that Tim will fail.'' \cr
\end{tabular}\\
Assume that if Carmine is the student that will pass, then Professor Gore tells Carmine randomly that Jeff will fail or that Tim will fail (with equal probabilities $1/2$).

Observe that $\pr{J} = \pr{T} = \pr{C} = 1/3$ and that $\pr{J' \given J} = 0$, $\pr{J' \given T} = 1$ and $\pr{J' \given C} = 1/2$. By Bayes's theorem, we have
\[
\pr{C \given J'} = \frac{\pr{C} \pr{J' \given C}}{\pr{J} \pr{J' \given J} + \pr{T} \pr{J' \given T} + \pr{C} \pr{J' \given C}} = \frac{1}{3}.
\]
%
\end{enumerate}
%End of Section C.2-----------------------------------------------------------------------------------------------------

%Section C.3------------------------------------------------------------------------------------------------------------
\section{Discrete random variables}
\subsection*{Notes}
\begin{enumerate}
%
\item Let $S$ be a sample space that is finite or countably infinite and $X$ be a random variable whose domain is $S$. Then
\[
\ex{X} = \sum_{s \in S} X(s) \pr{s}.
\]
\begin{proof}
Since
\[
\begin{array}{ll}
  & \ds{\sum_{s \in S} X(s) \pr{s}} \cr
= & \ds{\sum_{x \in \real} \sum_{s \in S \suchthat X(s) = x} X(s) \pr{s}} \cr
= & \ds{\sum_{x \in \real} x \multp \sum_{s \in S \suchthat X(s) = x} \pr{s}} \cr
= & \ds{\sum_{x \in \real} x \multp \pr{X = x}}. \cr
\end{array}
\]
\end{proof}
%
\end{enumerate}
\subsection*{Exercises}
\begin{enumerate}[\thesection-1]
%
\setcounter{enumi}{3}
%
\item It is useful to prove that \emph{for nonnegative random variables $Z$ and $W$ such that $Z(s) \leq W(s)$ for all $s \in S$ \paren{$S$ is the sample space}, it holds that $\ex{Z} \leq \ex{W}$} (or alternatively, that \emph{for nonnegative random variables $Z$, it holds that $\ex{Z} \geq 0$}).
%
\setcounter{enumi}{7}
%
\item It is useful to prove that \emph{for nonnegative random variables $Z$, it holds that $\ex{Z} \geq 0$}. Then, use (C.31).
%
\item Observe that $X$ is an indicator random variable and then apply Lemma 5.1.
%
\end{enumerate}
%End of Section C.3-----------------------------------------------------------------------------------------------------

%Section C.4------------------------------------------------------------------------------------------------------------
\section{The geometric and binomial distributions}
\subsection*{Exercises}
\begin{enumerate}[\thesection-1]
%
\setcounter{enumi}{4}
%
\item Skipped.
%
\item Since
\[
\lim_{n \to \infty} \parenf{1 + \frac{1}{n}}^n = \e.
\]
%
\item As an alternative method to prove
\[
\sum^n_{k = 0} {n \choose k}^2 = {2n \choose n},
\]
observe that
\[
(1 + x)^n = \sum^n_{k = 0} {n \choose k} x^k = \sum^n_{k = 0} {n \choose n - k} x^{n - k}
\]
and
\[
\sum^n_{k = 0} {n \choose k}^2 = \sum^n_{k = 0} {n \choose k} \multp {n \choose n - k}.
\]
%
\item Skipped.
%
\item Skipped.
%
\item Skipped.
%
\end{enumerate}
%End of Section C.4-----------------------------------------------------------------------------------------------------

%Section C.5------------------------------------------------------------------------------------------------------------
\section{The tails of the binomial distribution}
\subsection*{Exercises}
\begin{enumerate}[\thesection-1]
%
\item The former if $n > 1$.
%
\item Skipped.
%
\item Skipped.
%
\item Skipped.
%
\item Skipped.
%
\item Skipped.
%
\item Let us write $f(\alpha) \defas \mu \e^\alpha - \alpha r$ for $\alpha \in \real$. Note that $\lim\limits_{\alpha \to -\infty} f(\alpha) = \lim\limits_{\alpha \to \infty} = \infty$. Differentiating $f(\alpha)$ with respect to $\alpha$ and setting it to zero, i.e., $f'(\alpha) = 0$, yields $\alpha = \ln(r/\mu)$, a positive number because $r > \mu$ by premise in the statement of Theorem C.8 in textbook. Therefore, $f(\alpha)$ achieves its minimum at $\alpha = \ln(r/\mu)$, and hence so does $\exp(\mu \e^\alpha - \alpha r) = \exp(f(\alpha))$ because $\exp$ is strictly increasing.
%
\end{enumerate}
%End of Section C.5-----------------------------------------------------------------------------------------------------

%Problems of Appendix C-------------------------------------------------------------------------------------------------
\section*{Problems}
\begin{enumerate}[\thechapter-1]
%
\item Skipped.
%
\end{enumerate}
\end{appendices}
\end{document} 