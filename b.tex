\setcounter{chapter}{1}
\chapter{Sets, Etc.}
%Section B.1------------------------------------------------------------------------------------------------------------
\section{Sets}
\subsection*{Exercises}
\begin{enumerate}[\thesection-1]
%
\setcounter{enumi}{1}
%
\item Use induction on $n$ and observe, for example, that $A_1 \intsc A_2 \intsc \etc \intsc A_n \intsc A_{n + 1} = (A_1 \intsc A_2 \intsc \etc \intsc A_n) \intsc A_{n + 1}$.
%
\item Use induction on $n$ to prove the statement: \emph{For all sets $A_1, A_2, \etl, A_n$, it holds that
\[
\begin{array}{lll}
\card{A_1 \union A_2 \union \etc \union A_n} & = & \phantom{+} \card{A_1} + \card{A_2} + \etc + \card{A_n} \cr
 & & - \card{A_1 \intsc A_2} - \card{A_1 \intsc A_3} - \etc \cr
 & & + \card{A_1 \intsc A_2 \intsc A_3} + \etc \cr
 & & \multicolumn{1}{c}{\etv} \cr
 & & + (-1)^{n - 1} \card{A_1 \intsc A_2 \intsc \etc \intsc A_n}. \cr
\end{array}
\]}And observe that $A_1 \union A_2 \union \etc \union A_n \union A_{n + 1} = (A_1 \union A_2 \union \etc \union A_n) \union A_{n + 1}$.

Alternatively, argue that for every element $a \in A_1 \union A_2 \union \etc \union A_n$, it holds that $a$ is counted exactly once on the right-hand side. More specifically, if $a \in A_{i_1} \intsc A_{i_2} \intsc \etc \intsc A_{i_k}$ for $1 \leq i_1 < i_2 < \etc < i_k \leq n$, then the occurrences of $a$ on the right-hand side are counted up by $1$ for each of $\card{A_{i_1}}, \card{A_{i_2}}, \etl, \card{A_{i_k}}$, and are counted down by $1$ for each of $\card{A_{i_1} \intsc A_{i_2}}, \card{A_{i_1} \intsc A_{i_3}}, \etl$, and so on (and note that ${k \choose 1} - {k \choose 2} + \etc + (-1)^{k - 2 \multp \floor{k / 2} + 1} {k \choose k} = {k \choose 0} = 1$).
%
\end{enumerate}
%End of Section B.1-----------------------------------------------------------------------------------------------------

%Section B.2------------------------------------------------------------------------------------------------------------
\section{Relations}
\subsection*{Exercises}
\begin{enumerate}[\thesection-1]
%
\setcounter{enumi}{4}
%
\item No. The empty relation $R = \emptyset$ is symmetric and transitive but not reflexive.
%
\end{enumerate}
%End of Section B.2-----------------------------------------------------------------------------------------------------

%Section B.3------------------------------------------------------------------------------------------------------------
\section{Functions}
\subsection*{Exercises}
\begin{enumerate}[\thesection-1]
%
\item
\begin{enumerate}[(a)]
%%
\item Since $f(A) \subseteq B$, we have $\card{f(A)} \leq \card{B}$. If $f$ is injective, then $\card{A} = \card{f(A)}$ and hence $\card{A} \leq \card{B}$.
%%
\item If $f$ is surjective, then $\setm{f^\circ(b)}{b \in B}$ forms a partition of $A$, where $f^\circ(b) = \setm{a \in A}{f(a) = b}$ denotes the \emph{preimage} of $b$ under $f$. Thus,
\[
\begin{array}{llll}
\card{A} & =    & \ds{\sum\limits_{b \in B} \card{f^\circ(b)}} & \mathtext{(since the sets \mathmode{f^\circ(b)} are pairwise disjoint and their union is \mathmode{A})} \cr
         & \geq & \ds{\sum\limits_{b \in B} 1} & \mathtext{(since the sets \mathmode{f^\circ(b)} are nonempty)} \cr
         & =    & \card{B}. & \cr
\end{array}
\]
%%
\end{enumerate}
%
\setcounter{enumi}{2}
%
\item Let the inverse of a binary relation $R$ be defined by $\inv{R} = \setm{(b, a)}{(a, b) \in R}$.
%
\item The function $\inv{f} \cmps g \cmps f$ ($\cmps$ denotes \emph{functional composition}) is a bijection from $\zah$ to $\zah \cart \zah$, where
\[
f: \nat \to \zah, f(n) = (-1)^n \ceil{n/2}
\]
and
\[
\inv{f}: \zah \to \nat, f(m) =
\begin{cases}
2m      & \mathtext{if \mathmode{m \geq 0}}, \cr
-2m - 1 & \mathtext{if \mathmode{m < 0}} \cr
\end{cases}
\]
are the two bijections given in textbook, and where the bijection $g : \nat \to \nat \cart \nat$ is defined by
\[
g(m, n) = \frac{1}{2}(m + n)(m + n + 1) + m.
\]
%
\end{enumerate}
%End of Section B.3-----------------------------------------------------------------------------------------------------

%Section B.4------------------------------------------------------------------------------------------------------------
\section{Functions}
\subsection*{Exercises}
\begin{enumerate}[\thesection-1]
%
\item Every edge $(u, v)$ is counted twice on the left-hand side term: Once for $u$ and once for $v$.
%
\item A non-simple path (cycle) contains a vertex that occurs at least twice.
%
\item Use induction on the number of vertices $\card{V}$. In the inductive step, consider a distinguished vertex $v$ and the graph induced by $V - \sete{v}$.
%
\item Verification part skipped. Reflexivity and transitivity hold in the case of directed graphs.
%
\item Skipped.
%
\item Two vertices $u$ and $v$ are adjacent via a hyperedge if and only if they are adjacent to the same vertex representing this hyperedge in the corresponding bipartite graph.
%
\end{enumerate}
%End of Section B.4-----------------------------------------------------------------------------------------------------

%Section B.5------------------------------------------------------------------------------------------------------------
\section{Functions}
\subsection*{Exercises}
\begin{enumerate}[\thesection-1]
%
\item Skipped.
%
\item Attempted.
%
\item Skipped.
%
\item Skipped.
%
\item Attempted.
%
\item Attempted.
%
\item Attempted.
%
\end{enumerate}
%End of Section B.5-----------------------------------------------------------------------------------------------------

%Problems of Appendix B-------------------------------------------------------------------------------------------------
\section*{Problems}
\begin{enumerate}[\thechapter-1]
%
\item Skipped.
%
\end{enumerate}
%End of Problems of Appendix B------------------------------------------------------------------------------------------