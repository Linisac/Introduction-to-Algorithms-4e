\setcounter{chapter}{2}
\chapter{Counting and Probability}
%Section C.1-----------------------------------------------------------------------
\section{Counting}
\subsection*{Exercises}
\begin{enumerate}[\thesection-1]
%
\item For $1 \leq k \leq n$, the number of $k$-substrings of an $n$-string is $(n - k + 1)$; there is one $0$-substring of an $n$-string, namely the empty string. Therefore, an $n$-string has in total
\[
1 + \sum^n_{k = 1} (n - k + 1) = 1 + \frac{1}{2}n(n + 1)
\]
substrings.

Note that the number of nonempty substrings is equal to ${n + 1 \choose 2}$, and it can be ``explained'' by this counting argument: Every nonempty substring can be identified by the smallest and the greatest positions, namely two (not necessarily distinct) numbers $i$ and $j$ from the set $\sete{1, \etl, n}$; the total number of combinations of $i$ and $j$ is equal to the number of $2$-combinations of the set $\sete{1, \etl, n, \mbox{``duplicate''}}$, where $\sete{k, \mbox{``duplicate''}}$ is interpreted as $i = j = k$.
%
\setcounter{enumi}{4}
%
\item Consider the $k$-combinations of an $n$-set, in which exactly one of the $k$ chosen member is a designated \emph{leader}. Either
\begin{inparaenum}[(1)]
%%
\item first choose the $k$-combination (including the leader) and then choose one of the $k$ members as the leader, or
%%
\item first choose the leader and then choose the remaining $k - 1$ non-leader members.
%%
\end{inparaenum}
%
\item Consider the $(k + 1)$-combinations of an $n$-set, in which exactly one of the $k + 1$ chosen member is a designated \emph{leader}. Either
\begin{inparaenum}[(1)]
%%
\item first choose the $k$ non-leader members and then choose the leader from the remaining $n - k$ objects (not chosen earlier), or
%
\item first choose the leader and then choose the $k$ members from the remaining $n - 1$ objects.
%%
\end{inparaenum}
%
\setcounter{enumi}{8}
%
\item Consider the number of nonempty substrings of an $n$-string: Each substring can be identified by the leftmost position and the rightmost position.
%
\end{enumerate}
%End of Section C.1----------------------------------------------------------------
