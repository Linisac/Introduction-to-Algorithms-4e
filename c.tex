\setcounter{chapter}{2}
\chapter{Counting and Probability}
%Section C.1------------------------------------------------------------------------------------------------------------
\section{Counting}
\subsection*{Exercises}
\begin{enumerate}[\thesection-1]
%
\item For $1 \leq k \leq n$, the number of $k$-substrings of an $n$-string is $(n - k + 1)$; there is one $0$-substring of an $n$-string, namely the empty string. Therefore, an $n$-string has in total
\[
1 + \sum^n_{k = 1} (n - k + 1) = 1 + \frac{1}{2}n(n + 1)
\]
substrings.

Note that the number of nonempty substrings is equal to ${n + 1 \choose 2}$, and it can be ``explained'' by this counting argument: Every nonempty substring can be identified by the smallest and the greatest positions, namely two (not necessarily distinct) numbers $i$ and $j$ from the set $\sete{1, \etl, n}$; the total number of combinations of $i$ and $j$ is equal to the number of $2$-combinations of the set $\sete{1, \etl, n, \mbox{``duplicate''}}$, where $\sete{k, \mbox{``duplicate''}}$ is interpreted as $i = j = k$.
%
\setcounter{enumi}{4}
%
\item Consider the $k$-combinations of an $n$-set, in which exactly one of the $k$ chosen member is a designated \emph{leader}. Either
\begin{inparaenum}[(1)]
%%
\item first choose the $k$-combination (including the leader) and then choose one of the $k$ members as the leader, or
%%
\item first choose the leader and then choose the remaining $k - 1$ non-leader members.
%%
\end{inparaenum}
%
\item Consider the $(k + 1)$-combinations of an $n$-set, in which exactly one of the $k + 1$ chosen member is a designated \emph{leader}. Either
\begin{inparaenum}[(1)]
%%
\item first choose the $k$ non-leader members and then choose the leader from the remaining $n - k$ objects (not chosen earlier), or
%
\item first choose the leader and then choose the $k$ members from the remaining $n - 1$ objects.
%%
\end{inparaenum}
%
\setcounter{enumi}{8}
%
\item Consider the number of nonempty substrings of an $n$-string: Each substring can be identified by the leftmost position and the rightmost position.
%
\item Assume that $n \geq 2$. The expression ${n \choose k}$ achieves its locally maximum value when ${n \choose k} / {n \choose k - 1} \geq 1$ and ${n \choose k + 1} / {n \choose k} \geq 1$ hold simultaneously.
%
\item Combinatorial argument: choose $j + k$ items out of $n$ vs.~choose $j + k$ items out of $n$ in which $j$ are assigned a label.
%
\item Skipped.
%
\setcounter{enumi}{14}
%
\item Ignore $k = 0$ on the left hand side, i.e., assume $k \geq 1$. Form a nonempty group of people out of $n$, in which there is a \emph{leader}. The left side is calculated by the size of the group, while the right side is calculated by asking, after the leader has been chosen, whether each of the remaining $n - 1$ people is to be counted in or not.
%
\item Skipped.
%
\end{enumerate}
%End of Section C.1-----------------------------------------------------------------------------------------------------

%Section C.2------------------------------------------------------------------------------------------------------------
\section{Probability}
\subsection*{Exercises}
\begin{enumerate}[\thesection-1]
%
\setcounter{enumi}{1}
%
\item Let $A_1, A_2, \etl$ be a countably infinite sequence of events. Observe that
\begin{enumerate}[(1)]
%%
\item $\pr{A_{i + 1}} = \pr{(A_1 \union \etc \union A_i) \intsc A_{i + 1}} + \pr{(\cmpl{A_1} \intsc \etc \intsc \cmpl{A_i}) \intsc A_{i + 1}}$ for all $i \geq 1$ (by axiom 3),
%%
\item $A_1 \union A_2 \union \etc = A_1 \union (\cmpl{A_1} \intsc A_2) \union \etc$.
%%
\end{enumerate}
By (1) and axiom 1, we have that
\begin{enumerate}[(1)]
%
\setcounter{enumii}{2}
%%
\item $0 \leq \pr{(\cmpl{A_1} \intsc \etc \intsc \cmpl{A_{i - 1}}) \intsc A_i} \leq \pr{A_i}$ for all $i \geq 2$.
%%
\end{enumerate}
Thus,
\[
\begin{array}{lll}
     & \pr{A_1 \union A_2 \union \etc} & \cr
=    & \pr{A_1 \union (\cmpl{A_1} \intsc A_2) \union \etc} & \mbox{(by (2))} \cr
=    & \ds{\pr{A_1} + \sum^\infty_{i = 2} \pr{(\cmpl{A_1} \intsc \etc \intsc \cmpl{A_{i - 1}}) \intsc A_i}} & \mbox{(by axiom 3)} \cr
\leq & \ds{\pr{A_1} + \sum^\infty_{i = 2} \pr{A_i}} & \mbox{(by (3))} \cr
=    & \pr{A_1} + \pr{A_2} + \etc. & \cr
\end{array}
\]
%
\setcounter{enumi}{4}
%
\item By induction on $n \geq 2$.
%
\item Skipped.
%
\item Let $S = \sete{(0, 0, 0), (0, 1, 0), (1, 0, 0), (1, 1, 0), (0, 0, 1)}$, $A = \setm{(x, y, z)}{x = 0}$, $B = \setm{(x, y, z)}{y = 0}$ and $C = \setm{(x, y, z)}{z = 0}$.
%
\item Consider the following list of symbols for different events.\\
\begin{tabular}{lll}
$J$  & for & ``Jeff will pass,'' \cr
$T$  & for & ``Tim will pass,'' \cr
$C$  & for & ``Carmine will pass,'' \cr
$J'$ & for & ``Professor Gore tells Carmine that Jeff will fail,'' \cr
$T'$ & for & ``Professor Gore tells Carmine that Tim will fail.'' \cr
\end{tabular}\\
Assume that if Carmine is the student that will pass, then Professor Gore tells Carmine randomly that Jeff will fail or that Tim will fail (with equal probabilities $1/2$).

Observe that $\pr{J} = \pr{T} = \pr{C} = 1/3$ and that $\pr{J' \given J} = 0$, $\pr{J' \given T} = 1$ and $\pr{J' \given C} = 1/2$. By Bayes's theorem, we have
\[
\pr{C \given J'} = \frac{\pr{C} \pr{J' \given C}}{\pr{J} \pr{J' \given J} + \pr{T} \pr{J' \given T} + \pr{C} \pr{J' \given C}} = \frac{1}{3}.
\]
%
\end{enumerate}
%End of Section C.2-----------------------------------------------------------------------------------------------------

%Section C.3------------------------------------------------------------------------------------------------------------
\section{Discrete random variables}
\subsection*{Notes}
\begin{enumerate}
%
\item Let $S$ be a sample space that is finite or countably infinite and $X$ be a random variable whose domain is $S$. Then
\[
\ex{X} = \sum_{s \in S} X(s) \pr{s}.
\]
\begin{proof}
Since
\[
\begin{array}{ll}
  & \ds{\sum_{s \in S} X(s) \pr{s}} \cr
= & \ds{\sum_{x \in \real} \sum_{s \in S \suchthat X(s) = x} X(s) \pr{s}} \cr
= & \ds{\sum_{x \in \real} x \multp \sum_{s \in S \suchthat X(s) = x} \pr{s}} \cr
= & \ds{\sum_{x \in \real} x \multp \pr{X = x}}. \cr
\end{array}
\]
\end{proof}
%
\end{enumerate}
\subsection*{Exercises}
\begin{enumerate}[\thesection-1]
%
\setcounter{enumi}{3}
%
\item It is useful to prove that \emph{for nonnegative random variables $Z$ and $W$ such that $Z(s) \leq W(s)$ for all $s \in S$ \paren{$S$ is the sample space}, it holds that $\ex{Z} \leq \ex{W}$} (or alternatively, that \emph{for nonnegative random variables $Z$, it holds that $\ex{Z} \geq 0$}).
%
\setcounter{enumi}{7}
%
\item It is useful to prove that \emph{for nonnegative random variables $Z$, it holds that $\ex{Z} \geq 0$}. Then, use (C.31).
%
\item Observe that $X$ is an indicator random variable and then apply Lemma 5.1.
%
\end{enumerate}
%End of Section C.3-----------------------------------------------------------------------------------------------------

%Section C.4------------------------------------------------------------------------------------------------------------
\section{The geometric and binomial distributions}
\subsection*{Exercises}
\begin{enumerate}[\thesection-1]
%
\setcounter{enumi}{4}
%
\item Skipped.
%
\item Since
\[
\lim_{n \to \infty} \parenf{1 + \frac{1}{n}}^n = \e.
\]
%
\item As an alternative method to prove
\[
\sum^n_{k = 0} {n \choose k}^2 = {2n \choose n},
\]
observe that
\[
(1 + x)^n = \sum^n_{k = 0} {n \choose k} x^k = \sum^n_{k = 0} {n \choose n - k} x^{n - k}
\]
and
\[
\sum^n_{k = 0} {n \choose k}^2 = \sum^n_{k = 0} {n \choose k} \multp {n \choose n - k}.
\]
%
\item Skipped.
%
\item Skipped.
%
\item Skipped.
%
\end{enumerate}
%End of Section C.4-----------------------------------------------------------------------------------------------------

%Section C.5------------------------------------------------------------------------------------------------------------
\section{The tails of the binomial distribution}
\subsection*{Exercises}
\begin{enumerate}[\thesection-1]
%
\item The former if $n > 1$.
%
\item HERE.
%
\end{enumerate}
%End of Section C.5-----------------------------------------------------------------------------------------------------

%Problems of Appendix C-------------------------------------------------------------------------------------------------
\section*{Problems}
\begin{enumerate}[\thechapter-1]
%
\item Skipped.
%
\end{enumerate}